\chapter{Desarrollo del Proyecto}\label{chap:desarrollo}
\section{Descripción del Problema}

El problema central de este trabajo es la autenticación continua y pasiva de usuarios en dispositivos móviles 
a partir de su biometría conductual. A diferencia de los mecanismos de autenticación estática —contraseñas, 
PINs o biometría fisiológica— que únicamente verifican la identidad en el momento del acceso, el objetivo es 
mantener una verificación activa durante toda la sesión sin que el usuario deba realizar ninguna acción adicional.
Este enfoque plantea un problema de \textbf{verificación biométrica de una clase} (\textit{one-class} o \textit{open-set}): 
dado un perfil conductual enrolado para un usuario legítimo, el sistema debe decidir, a partir de una ventana 
corta de interacción, si el individuo que está operando el dispositivo corresponde a dicho perfil o se trata 
de un impostor. La dificultad inherente reside en que durante la fase de entrenamiento no se dispone de muestras 
etiquetadas de todos los posibles impostores, lo que impide formular el problema como una clasificación cerrada 
convencional.
Las modalidades conductuales seleccionadas para este trabajo son tres, elegidas por su complementariedad y por
la riqueza de información identitaria que aportan:

\begin{itemize}
    \item \textbf{Dinámica de tecleo (\textit{keystroke dynamics})}: secuencias de tiempos de pulsación (\textit{Dwell Time}, DT),
     tiempos de vuelo entre teclas (\textit{Flight Time}, FT) e intervalos entre pulsaciones consecutivas 
     (\textit{Inter-key Interval}, IKI). Constituye la modalidad más estudiada en la literatura y ofrece alta 
     discriminabilidad incluso con ráfagas cortas de texto.
    \item \textbf{Dinámica de firma (\textit{signature dynamics})}: patrones de presión, velocidad y
     aceleración durante el trazo de la firma en pantalla táctil. Aporta una señal de alta varianza 
     inter-usuario y baja varianza intra-usuario cuando el individuo firma de forma natural.
    \item \textbf{Gestos de desplazamiento (\textit{scroll dynamics})}: características extraídas de 
    los movimientos de deslizamiento sobre la pantalla, incluyendo velocidad, aceleración, longitud del trazo 
    y ángulo de desplazamiento. Su inclusión está condicionada al análisis del beneficio marginal que aporta en 
    la fusión multimodal.
\end{itemize}

El principal desafío técnico es la \textbf{heterogeneidad temporal} de las señales: la dinámica de tecleo 
es discreta y de longitud variable, la firma es una señal continua de corta duración, y el scroll produce 
eventos esporádicos dependientes del contexto de uso. El sistema debe ser capaz de integrar estas tres fuentes 
de forma coherente y tomar decisiones de autenticación con latencia mínima, idealmente tras menos de 50 eventos
de interacción.

\section{Metodología}\label{sec:metodologia}
Para abordar el problema descrito, se propone una metodología experimental comparativa basada en dos 
enfoques arquitectónicos complementarios, cuyo rendimiento se evaluará bajo un protocolo común.

\subsection{Preprocesamiento y Extracción de Características}
Previo al modelado, cada modalidad requiere un pipeline de preprocesamiento específico:

\begin{itemize}
    \item \textbf{Tecleo}: a partir de los eventos de pulsación se extraen las tripletas $(DT_i, FT_i, IKI_i)$ para cada tecla 
    $i$ de la secuencia. Las secuencias se segmentan en ventanas deslizantes de longitud fija $N$ 
    (objetivo: $N = 50$ eventos), con solapamiento configurable para la autenticación continua. Los 
    valores atípicos se filtran mediante umbrales basados en percentiles y las secuencias se normalizan
     con estadísticos calculados por usuario en la fase de enrolamiento.
    \item \textbf{Firma}: se extraen series temporales de posición $(x_t, y_t)$, presión $p_t$ y sus derivadas 
    de primer y segundo orden. La señal se remuestrea a una frecuencia fija para garantizar la homogeneidad 
    de la representación de entrada.
    \item \textbf{Scroll}: se caracterizan cada gesto de deslizamiento mediante un vector de rasgos agregados: 
    velocidad media, velocidad máxima, aceleración media, longitud euclidiana del trazo, duración y desviación 
    angular respecto al eje vertical.
\end{itemize}

\subsection{Enfoque 1: Arquitectura basada en Transformer}

El primer enfoque propone una arquitectura \textit{end-to-end} inspirada en BehaveFormer~\cite{senerath2023behaveformer}
 y Type2Branch~\cite{gonzalez2025type2branch}, que procesa las tres modalidades de forma conjunta 
 mediante mecanismos de atención.
Cada secuencia de interacción se trata como una serie de \textit{tokens} de entrada. Un codificador 
Transformer independiente por modalidad —con capas de \textit{Multi-Head Self-Attention} y redes 
\textit{feed-forward} posicionales— genera una representación de contexto para cada fuente de señal. Las 
representaciones resultantes se fusionan mediante un módulo de \textbf{atención cruzada entre modalidades} 
(\textit{cross-modal attention}), que aprende a ponderar dinámicamente la contribución de cada fuente en función del contexto de uso.
La función de pérdida empleada para el entrenamiento es \textit{Set2set Loss}~\cite{gonzalez2025type2branch}, 
que compara un conjunto de muestras de enrolamiento $\mathcal{S} = \{s_1, \ldots, s_k\}$ contra una muestra de consulta $q$, 
permitiendo una estimación más robusta de la similitud que el clásico \textit{Triplet Loss} puntual:

\begin{equation}
    \mathcal{L}_{\text{set2set}} = \max\left(0,\; m + d\!\left(\bar{e}_{\mathcal{S}},\, q^+\right) - d\!\left(\bar{e}_{\mathcal{S}},\, q^-\right)\right)
    \label{eq:set2set}
\end{equation}

donde $\bar{e}_{\mathcal{S}}$ es la representación agregada del conjunto de enrolamiento, 
$q^+$ una muestra del usuario legítimo, $q^-$ una muestra de impostor y $m$ el margen de separación.

\subsection{Enfoque 2: Ensemble de Modelos Especializados por Modalidad}

El segundo enfoque adopta una estrategia modular: se entrena un modelo especializado e independiente 
para cada modalidad conductual, y sus puntuaciones de similitud se combinan mediante una estrategia de fusión 
a nivel de \textit{score}.

\begin{itemize}
    \item \textbf{Modelo de tecleo}: red LSTM bidireccional seguida de una capa de proyección métrica, 
    entrenada con \textit{Triplet Loss}, siguiendo la línea de TypeNet~\cite{acien2022typenet}.
    \item \textbf{Modelo de firma}: red CNN-LSTM que extrae características locales de los segmentos del 
    trazo (CNN) y modela la dinámica temporal global (LSTM).
    \item \textbf{Modelo de scroll}: clasificador ligero basado en un perceptrón multicapa (MLP) o una red 
    CNN-1D sobre los vectores de rasgos agregados por gesto.
\end{itemize}

La fusión de puntuaciones se explorará mediante tres estrategias: combinación lineal con pesos fijos, 
combinación lineal con pesos aprendidos, y un meta-clasificador entrenado sobre las puntuaciones de los 
modelos base.

\subsection{Protocolo de Evaluación}

Ambos enfoques se evaluarán bajo un protocolo común de verificación, siguiendo las directrices establecidas en 
las competiciones MobileB2C~\cite{stragapede2022mobileb2c}:

\begin{itemize}
    \item \textbf{Partición de datos}: división en conjuntos de enrolamiento, desarrollo y evaluación, 
    sin solapamiento de usuarios entre particiones.
    \item \textbf{Escenarios de impostor}: se considerarán impostores aleatorios (usuarios distintos sin 
    conocimiento del objetivo) e impostores hábiles (usuarios que intentan imitar al legítimo), cuando el 
    dataset seleccionado lo permita.
    \item \textbf{Métricas principales}: \textit{Equal Error Rate} (EER) como métrica primaria de comparación, 
    complementada con AUC, FAR@1\%FRR y el \textit{Time-to-Detection} (TTD) estimado.
\end{itemize}


\section{Resultados Preliminares}\label{sec:resultados_preliminares}

En el momento de redacción de este documento, el desarrollo del sistema se encuentra en fase de diseño 
arquitectónico y planificación experimental. Por tanto, los resultados que se presentan a continuación son 
estimaciones de referencia (\textit{baseline}) derivadas de la replicación parcial de trabajos del estado del 
arte sobre datasets públicos, con el objetivo de establecer un punto de comparación para las arquitecturas 
propuestas.

\subsection{Baseline de Referencia}

A continuación se muestra la tabla que recoge los valores de EER reportados por los sistemas más relevantes del estado
 del arte sobre las modalidades y datasets que se emplearán en este trabajo,
  y que constituyen el umbral de rendimiento que el sistema propuesto deberá superar.

\begin{table}[h]
\centering
\caption{Resultados de referencia del estado del arte (EER~\%).}\label{tab:baseline}
\begin{tabular}{llccc}
\textbf{Sistema} & \textbf{Modalidad} & \textbf{Dataset} & \textbf{EER (\%)} & \textbf{Seq. length} \\
TypeNet~\cite{acien2022typenet}             & Tecleo              & Aalto DB   & 2.20 & 50 chars \\
Type2Branch~\cite{gonzalez2025type2branch}  & Tecleo              & Mobile     & 1.03 & 50 chars \\
BehaveFormer~\cite{senerath2023behaveformer}& Tecleo + IMU        & HuMIdb     & 2.95 & ---      \\
MobileB2C top-1~\cite{stragapede2022mobileb2c} & Keystroke      & BehavePassDB & \multicolumn{1}{c}{AUC: 66.37} & --- \\
\end{tabular}
\end{table}

\subsection{Estimaciones Esperadas del Sistema Propuesto}

A modo orientativo, y en base a las mejoras arquitectónicas propuestas respecto a los sistemas de referencia,
 se anticipan los rangos de rendimiento recogidos en la siguiente tabla:

\begin{table}[h]
\centering
\caption{Estimación de rendimiento esperado del sistema propuesto (EER~\%). 
\textbf{[PENDIENTE DE ACTUALIZAR]}}\label{tab:resultados_esperados}
\begin{tabular}{llcc}
\textbf{Enfoque} & \textbf{Modalidades} & \textbf{EER esperado (\%)} & \textbf{Observaciones} \\
Transformer unificado  & Tecleo + Firma          & $\sim$1.5 -- 2.5  & Fusión cross-modal \\
Transformer unificado  & Tecleo + Firma + Scroll & $\sim$1.2 -- 2.0  & Mejora con scroll  \\
Ensemble modular       & Tecleo + Firma          & $\sim$1.8 -- 3.0  & Fusión de scores   \\
Ensemble modular       & Tecleo + Firma + Scroll & $\sim$1.5 -- 2.5  & Meta-clasificador  \\
\end{tabular}
\end{table}

Estas estimaciones se fundamentan en la hipótesis de que la fusión multimodal de tecleo y firma 
aportará una mejora consistente respecto al uso de tecleo en solitario, tal y como sugieren los trabajos 
previos en fusión de modalidades conductuales~\cite{stragapede2022patterns}. 
La incorporación del scroll se plantea como un experimento ablativo para cuantificar su contribución marginal 
al rendimiento global del sistema.