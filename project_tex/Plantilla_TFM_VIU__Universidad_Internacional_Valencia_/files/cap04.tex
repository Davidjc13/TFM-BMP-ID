\section{Sistemas de autenticación biométrica}

La biometría se ha consolidado como el pilar fundamental de la identidad digital moderna, 
desplazando a los sistemas basados en posesión y conocimiento. 
Este marco teórico analiza la biometría no solo como una herramienta técnica, sino como un proceso científico de reconocimiento 
de individuos basado en sus características biológicas y conductuales.

La biometría se basa en la premisa de que ciertos rasgos físicos o de comportamiento son únicos, estables y medibles.
Para que una característica humana sea considerada un indicador biométrico eficaz, debe cumplir con siete criterios fundamentales 
definidos por \cite{jain2008handbookofbiometrics} (Sección 1.6 \textit{biometric characteristics}):

\begin{enumerate}
    \item \textbf{Universalidad}: Cada persona debe poseer el rasgo.
    \item \textbf{Unicidad}: Dos personas no deben tener el mismo rasgo (distinguibilidad).
    \item \textbf{Permanencia}: El rasgo debe ser \textit{invariante} en el tiempo.
    \item \textbf{Colectabilidad}: El rasgo debe poder medirse cuantitativamente.
    \item \textbf{Rendimiento}: Precisión y velocidad en el reconocimiento.
    \item \textbf{Aceptabilidad}: Grado en que la población está dispuesta a utilizar el sistema.
    \item \textbf{Resistencia al fraude}: Dificultad para engañar al sistema mediante artefactos.
\end{enumerate}

Los sistemas biométricos se basan en un flujo de trabajo estándar dividido en dos fases críticas:

\begin{enumerate}
    \item \textbf{Alistamiento o Registro}: En esta fase un sistema de sensores captura la huella biométrica del individuo. Posteriormente se elimina el ruido y se extraen las características de interés de la huella. Finalmente se genera una representación matemática de la huella. Es importante remarcar que no se guarda la imagen original de la huella sino su representación matemática.
    \item \textbf{Reconocimiento}: En esta fase se distinguen dos procesos:
    \begin{itemize}
        \item \textbf{Verificación}: Se realiza la comparación \textit{uno a uno} de la huella (\textit{¿Es, realmente, la persona que dice ser?}).
        \item \textbf{Identificación}: Se realiza la comparación \textit{uno a varios} de la huella (\textit{¿Quién es esta persona?}).
    \end{itemize}
\end{enumerate}

(Incluir un diagrama de los dos sistemas quedaría bien)

\subsection*{Modalidades de biometría conductual}

En la literatura científica es común ver que la biometría se clasifica en tres grupos:

\begin{enumerate}
    \item \textbf{Biometría Fisiológica}: Se centra en la medida directa de diferentes partes del cuerpo como: la huella dactilar, el reconocimiento facial, ocular, geometría de la mano y su patrón de venas.
    \item \textbf{Biometría Conductual}: Analiza patrones de comportamientos que pueden ser aprendidos o adquiridos como: la mecanografía, la marcha (forma de caminar) y la firma.
    \item \textbf{Biometría Química}: Se basa en las características químicas del individuo como el ADN o el olor corporal (compuestos órganicos volátiles emitidos por la piel).
\end{enumerate}

En este trabajo nos centraremos en la biometría conductual o del comportamiento.

\subsection*{Métricas de Rendimiento y Errores}

Para medir la eficacia del sistema biométrico usaremos principalmente las siguientes tasas de error fundamentales:
\begin{enumerate}
    \item \textbf{FAR (\textit{False Acceptance Rate})}: Probabilidad de que el sistema acepte por error a un impostor.
    \item \textbf{FRR (\textit{False Rejection Rate})}: Probabilidad de que el sistema rechace erróneamente a un usuario legítimo.
    \item \textbf{ERR (\textit{Equal Error Rate})}: El punto donde \textit{FAR} y \textit{FRR} son iguales.
\end{enumerate}

\section{Métodos de aprendizaje automático aplicados}

La evolución de la biometría moderna está intrínsecamente ligada al avance del \textit{Aprendizaje Automático} y del \textit{Aprendijaze Profundo}. Estas arquitecturas permiten pasar de una comparación basada
en reglas manuales a una extracción de características latentes con dimensionalidad alta.

\subsection{Redes de Memoria a Largo Plazo}

Las redes neuronales LSTM \textit{(Long Short-Term Memory)} son una variante 
de las Redes Neuronales Recurrentes \textit{(RNN)} diseñadas para evitar el 
problema del desvanecimiento del gradiente \textit{(vanishing gradient)}, 
permitiendo aprender dependencias a largo plazo en datos secuenciales 
\cite{hochreiter1997lstm}.

\subsubsection{Arquitectura y Mecanismo de Compuertas}

A diferencia de una RNN estándar, cuya celda oculta se actualiza 
completamente en cada paso temporal, la LSTM introduce un estado de celda 
$C_t$ que actúa como una ``memoria'' capaz de retener información relevante 
durante largas secuencias. Este mecanismo se regula mediante tres compuertas 
diferenciables:

\begin{itemize}
    \item \textbf{Compuerta de olvido} $f_t$: Decide qué información del 
    estado anterior debe descartarse. Se define como:
    \begin{equation}
        f_t = \sigma(W_f \cdot [h_{t-1}, x_t] + b_f)
    \end{equation}

    \item \textbf{Compuerta de entrada} $i_t$: Controla qué nueva información 
    se almacena en el estado de celda:
    \begin{equation}
        i_t = \sigma(W_i \cdot [h_{t-1}, x_t] + b_i)
    \end{equation}
    \begin{equation}
        \tilde{C}_t = \tanh(W_C \cdot [h_{t-1}, x_t] + b_C)
    \end{equation}

    \item \textbf{Compuerta de salida} $o_t$: Determina qué parte del estado 
    de celda se expone como salida:
    \begin{equation}
        o_t = \sigma(W_o \cdot [h_{t-1}, x_t] + b_o)
    \end{equation}
    \begin{equation}
        h_t = o_t \cdot \tanh(C_t)
    \end{equation}
\end{itemize}

Donde $\sigma$ es la función sigmoide, $W$ son las matrices de pesos 
entrenables, $b$ los sesgos, $h_{t-1}$ el estado oculto anterior y $x_t$ 
la entrada en el instante $t$. La actualización del estado de celda combina 
el olvido selectivo del pasado con la incorporación de nueva información:

\begin{equation}
    C_t = f_t \odot C_{t-1} + i_t \odot \tilde{C}_t
\end{equation}

donde $\odot$ denota el producto elemento a elemento \textit{(Hadamard)}.

\subsection{Ventajas frente a RNN's clásicas}

El problema del desvanecimiento del gradiente en RNN estándar surge cuando, 
durante la retropropagación a través del tiempo \textit{(BPTT)}, el gradiente 
se multiplica repetidamente por valores menores que uno, haciendo que la señal 
de error se extinga antes de alcanzar pasos temporales lejanos. Las LSTM 
mitigan este efecto gracias al estado de celda, que permite que el gradiente 
fluya sin atenuación a través de largos intervalos temporales siempre que la 
compuerta de olvido permanezca abierta \cite{goodfellow2016deep}.

\subsubsection{Aplicación a la Biometría Conductual}

En el contexto de la autenticación basada en biometría del comportamiento, 
las LSTM resultan especialmente adecuadas por la naturaleza inherentemente 
secuencial de las señales capturadas. Para la \textbf{dinámica de tecleo}, 
cada pulsación genera un vector de características temporales —Dwell Time 
($DT$), Flight Time ($FT$) e Inter-key Interval ($IKI$)— que forman una 
secuencia variable en longitud. La LSTM es capaz de modelar las dependencias 
entre pulsaciones no adyacentes, capturando el ritmo global de escritura 
del usuario, algo imposible de representar con modelos estáticos como SVM 
o KNN \cite{acien2022keystroke}.

Del mismo modo, para señales de \textbf{sensores inerciales} (acelerómetro 
y giroscopio), la LSTM aprende patrones de micro-gestualidad que se 
manifiestan en ventanas temporales de cientos de milisegundos, extrayendo 
representaciones del comportamiento físico del usuario al sostener e 
interactuar con el dispositivo.

Trabajos como TypeNet \cite{acien2022keystroke} y BehaveFormer 
\cite{senerath2023behaveformer} han demostrado que las LSTM alcanzan 
resultados competitivos en autenticación continua de texto libre, siendo 
posteriormente complementadas con mecanismos de atención y arquitecturas 
basadas en Transformers para capturar dependencias tanto locales como globales 
en la secuencia de interacción.

\subsection{Transformers y Mecanismos de Atención}

Originalmente propuestos por \citeauthor{vaswani2017attention} para el 
procesamiento del lenguaje natural, los Transformers han revolucionado 
múltiples campos de la inteligencia artificial gracias a su capacidad para 
modelar relaciones globales entre todos los elementos de una secuencia de 
forma paralela, superando las limitaciones de las arquitecturas recurrentes 
en cuanto a eficiencia computacional y captura de dependencias a larga 
distancia \cite{vaswani2017attention}.

\subsubsection{El Mecanismo de Atención}

El núcleo del Transformer es el mecanismo de \textit{Scaled Dot-Product 
Attention}, que opera sobre tres matrices: consultas \textit{(Queries, $Q$)}, 
claves \textit{(Keys, $K$)} y valores \textit{(Values, $V$)}, todos ellos 
proyecciones lineales de la entrada. La atención se calcula como:

\begin{equation}
    \text{Attention}(Q, K, V) = \text{softmax}\left(\frac{QK^T}{\sqrt{d_k}}\right)V
\end{equation}

donde $d_k$ es la dimensión de las claves, y el factor $\frac{1}{\sqrt{d_k}}$ 
evita que el producto escalar crezca excesivamente para dimensiones altas, 
estabilizando el gradiente durante el entrenamiento \cite{vaswani2017attention}.

En la práctica se emplea \textit{Multi-Head Attention}, que aplica $h$ 
cabezas de atención en paralelo sobre distintas proyecciones del espacio de 
representación, permitiendo al modelo atender simultáneamente a diferentes 
aspectos de la secuencia:

\begin{equation}
    \text{MultiHead}(Q, K, V) = \text{Concat}(\text{head}_1, \ldots, 
    \text{head}_h)W^O
\end{equation}
\begin{equation}
    \text{head}_i = \text{Attention}(QW_i^Q, KW_i^K, VW_i^V)
\end{equation}

\subsubsection{Atención Dual Espacio-Temporal en Biometría Conductual}

En el contexto de la autenticación basada en biometría del comportamiento, 
el mecanismo de atención adquiere una relevancia especial, ya que no todas 
las pulsaciones o instantes temporales contribuyen por igual a la identidad 
del usuario. Arquitecturas como BehaveFormer \cite{senerath2023behaveformer} 
introducen el \textit{Spatio-Temporal Dual Attention Transformer (STDAT)}, 
que aplica dos dimensiones de atención de forma simultánea:

\begin{itemize}
    \item \textbf{Atención Temporal}: Pondera la relevancia de cada instante 
    en la secuencia de pulsaciones, identificando qué momentos del flujo de 
    tecleo son más discriminativos para verificar la identidad.
    
    \item \textbf{Atención de Canal}: Opera sobre las características 
    extraídas de los sensores inerciales (IMU), determinando qué ejes del 
    acelerómetro o giroscopio aportan mayor información identitaria en cada 
    contexto de uso.
\end{itemize}

\subsubsection{Fusión Multimodal mediante Atención}

Una de las ventajas más relevantes de los Transformers para este trabajo es 
su capacidad natural para la \textbf{fusión multimodal}. Al tratar cada 
modalidad —dinámica de tecleo, gestos táctiles y señales IMU— como una 
secuencia de tokens, el mecanismo de atención aprende automáticamente a 
qué fuente de información debe dar mayor peso en cada instante, optimizando 
la decisión de autenticación sin necesidad de definir manualmente reglas de 
fusión \cite{senerath2023behaveformer}. Esto contrasta con los enfoques 
clásicos de fusión a nivel de puntuación, donde la combinación de modalidades 
se realiza mediante reglas heurísticas fijas.

Type2Branch \cite{gonzalez2025type2branch} extiende este principio 
incorporando capas de atención sobre una arquitectura dual CNN-LSTM, 
permitiendo al modelo ponderar eventos de pulsación específicos con mayor 
carga discriminativa, lo que se traduce en mejoras significativas del 
Equal Error Rate (EER) frente a modelos precedentes como TypeNet.

\section{Bases de datos y benchmarks del dominio}

Aquí incluiré lo que utilice para comparar más adelante...

\section{Privacidad y Consideraciones Éticas}

El desarrollo de sistemas de autenticación basados en biometría conductual 
implica la captura, procesamiento y almacenamiento de datos de carácter 
personal de especial sensibilidad. En consecuencia, cualquier implementación 
debe enmarcarse en el conjunto normativo vigente en materia de protección 
de datos, tanto a nivel europeo como nacional.

\subsection{Marco Normativo}

\subsubsection{Reglamento General de Protección de Datos (RGPD)}

El Reglamento (UE) 2016/679 del Parlamento Europeo y del Consejo, 
conocido como RGPD \cite{rgpd2016}, establece el marco general de 
protección de datos personales en la Unión Europea. En su artículo 9, 
el RGPD clasifica los \textbf{datos biométricos} como una categoría 
especial de datos personales, cuyo tratamiento queda prohibido con 
carácter general salvo que concurra alguna de las excepciones tasadas, 
entre las que destacan:

\begin{itemize}
    \item El \textbf{consentimiento explícito} del interesado para uno 
    o varios fines específicos.
    \item La necesidad del tratamiento por razones de \textbf{interés 
    público esencial}.
    \item Fines de \textbf{investigación científica}, siempre que se 
    apliquen las garantías adecuadas.
\end{itemize}

Adicionalmente, el RGPD consagra una serie de principios de obligado 
cumplimiento para cualquier sistema que trate datos biométricos 
conductuales:

\begin{itemize}
    \item \textbf{Minimización de datos}: únicamente deben recogerse 
    los datos estrictamente necesarios para el fin declarado. En el 
    contexto de este sistema, esto implica limitar la captura a las 
    métricas temporales de tecleo (DT, FT, IKI) y señales IMU 
    imprescindibles, evitando el almacenamiento del contenido textual 
    introducido por el usuario.

    \item \textbf{Limitación de la finalidad}: los datos recogidos no 
    podrán utilizarse para fines distintos a la autenticación del 
    usuario para el que fueron enrolados.

    \item \textbf{Exactitud y actualización}: los perfiles biométricos 
    conductuales deben mantenerse actualizados, dado que los patrones 
    de comportamiento del usuario pueden evolucionar con el tiempo 
    debido a factores como el estado emocional, la fatiga o el cambio 
    de dispositivo.

    \item \textbf{Limitación del plazo de conservación}: los datos 
    biométricos no deben conservarse más tiempo del necesario para 
    cumplir la finalidad del tratamiento.

    \item \textbf{Integridad y confidencialidad}: el sistema debe 
    garantizar la seguridad de los datos mediante cifrado y controles 
    de acceso adecuados.
\end{itemize}

\subsubsection{Ley Orgánica de Protección de Datos y Garantía de los 
Derechos Digitales (LOPDGDD)}

En el ámbito nacional español, la Ley Orgánica 3/2018, de 5 de 
diciembre, de Protección de Datos Personales y Garantía de los Derechos 
Digitales \cite{lopdgdd2018} complementa y adapta el RGPD al 
ordenamiento jurídico español. En materia de datos biométricos, la 
LOPDGDD refuerza las obligaciones del responsable del tratamiento y 
establece garantías adicionales, entre las que destacan:
\begin{itemize}
    \item La obligatoriedad de realizar una \textbf{Evaluación de 
    Impacto relativa a la Protección de Datos (EIPD)} antes de 
    implementar cualquier sistema que trate datos biométricos a gran 
    escala, tal y como exige el artículo 35 del RGPD.

    \item El reconocimiento explícito del \textbf{derecho a la 
    portabilidad} de los perfiles biométricos, permitiendo al usuario 
    solicitar la transferencia o eliminación de su perfil conductual 
    en cualquier momento.

    \item La figura del \textbf{Delegado de Protección de Datos (DPD)}, 
    cuya designación puede ser obligatoria cuando el tratamiento de 
    datos biométricos se realice a gran escala.
\end{itemize}

\subsection{Consideraciones Éticas del Sistema}

Más allá del cumplimiento normativo estricto, el diseño de un sistema 
de autenticación conductual plantea una serie de consideraciones éticas 
que deben abordarse desde las fases tempranas del desarrollo.

\subsubsection{Transparencia y Consentimiento Informado}

Dado que la biometría conductual opera de forma pasiva y transparente 
para el usuario, existe el riesgo de que este no sea plenamente 
consciente de los datos que se están recopilando. Por ello, es 
imprescindible garantizar que el usuario reciba una información clara, 
comprensible y accesible sobre el funcionamiento del sistema antes de 
otorgar su consentimiento, evitando el uso de cláusulas genéricas o 
formulaciones técnicas ininteligibles.

\subsubsection{Privacidad desde el Diseño}

El principio de \textit{Privacy by Design} \cite{cavoukian2009privacy}, 
incorporado al artículo 25 del RGPD como \textit{protección de datos 
desde el diseño y por defecto}, exige que las medidas de privacidad 
se integren en la arquitectura del sistema desde su concepción, y no 
como un añadido posterior. En la práctica, esto se traduce en decisiones 
de diseño como:

\begin{itemize}
    \item Almacenar únicamente representaciones vectoriales cifradas 
    del perfil conductual, nunca las secuencias de pulsaciones en bruto.
    \item Aplicar técnicas de \textbf{anonimización} o 
    \textbf{seudonimización} que impidan vincular el perfil biométrico 
    con la identidad real del usuario en caso de brecha de seguridad.
    \item Garantizar que el modelo entrenado no permita reconstruir el 
    contenido textual original a partir de las métricas temporales 
    capturadas.
\end{itemize}

\subsubsection{Sesgos y Equidad del Sistema}

Los modelos de aprendizaje profundo pueden presentar sesgos en su 
rendimiento en función de características del usuario como la edad, 
condición motora o nivel de familiaridad con el dispositivo. Un sistema 
de autenticación conductual que presente tasas de error 
significativamente distintas entre grupos de usuarios podría generar 
situaciones de discriminación inadvertida. Por ello, es necesario 
evaluar el rendimiento del sistema de forma estratificada y adoptar 
medidas correctoras cuando se detecten disparidades.

\subsubsection{Revocabilidad del Perfil Biométrico}

A diferencia de la biometría fisiológica, los patrones conductuales 
son, en cierta medida, revocables: si el perfil de un usuario se ve 
comprometido, es posible reentrenar el modelo con nuevas muestras. 
No obstante, esto implica mantener protocolos claros de gestión del 
ciclo de vida del perfil biométrico, incluyendo procedimientos de 
re-enrolamiento y eliminación segura de perfiles obsoletos.