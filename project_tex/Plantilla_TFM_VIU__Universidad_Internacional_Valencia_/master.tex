% PLANTILLA LaTeX VIU - mayo 2020
% Basado en plantilla original de Rafael Padilla (2011)
% Modificaciones de Juande Santander-Vela (2020)
%----------------------------------------------------------
% PREAMBULO:
\documentclass[a4paper,11pt]{book}
%----------------------------------------------------------
% PAQUETES:
% Language

% In moder computers, this should be set to UTF8
\usepackage[utf8]{inputenc}
%\usepackage[latin1]{inputenc} // only for non-UTF supporting 
\usepackage[spanish]{babel}

% SECTIONS WITH SmallCaps
\usepackage[T1]{fontenc}

% PAQUETE PARA AÑADIR URLs
\usepackage{url}
%% Define a new 'leo' style for the package that will use a smaller font.
\makeatletter
\def\url@leostyle{%
  \@ifundefined{selectfont}{\def\UrlFont{\sf}}{\def\UrlFont{\small\ttfamily}}}
\makeatother
%% Now actually use the newly defined style.
\urlstyle{leo}

% PARA HACER CLICKABLES LAS URLs EN EL PDF
\usepackage{hyperref}
% SETUP COLORES PDF
\hypersetup{
    colorlinks = true,
    linkcolor = black,
    anchorcolor = black,
    citecolor = black,
    filecolor = black,
    urlcolor = black
}

% Formulas y simbolos de matematicas:
\usepackage{amsmath}
\usepackage{amssymb}
\usepackage{gensymb}
\usepackage{wasysym}
\usepackage{mathrsfs}

% Graficos
\usepackage{graphicx}
\usepackage{float}
\usepackage{lscape}
\usepackage{wrapfig}

% NOTA:
% Si compilamos en DVI->dvips->ps2pdf tendremos que usar: EPS o PS (tamin BMP)
% Si compilamos directamente en pdf: JPEG o PNG

% INDEX
\usepackage{makeidx}

% MOSTRAR BIBLIOGRAFIA EN INDICE
\usepackage[nottoc]{tocbibind}

% PAQUETE PARA LA BIBLIOGRAFIA: 
% BibTeX
\usepackage[sectionbib]{natbib}  % Cross-reference package (Natural BiB)
\bibpunct{(}{)}{;}{a}{}{,}
%\usepackage{chapterbib}  % Put References at the end of each chapter
% Archivo con la bibliografia: ./files/TFM/Biblio/Bib.bib
% Compilar: Normal, BibTeX y 2 veces mas como Normal

% Tablas
\usepackage{multirow} % Para unir varias filas
\usepackage{longtable} % Para crear tablas de mas de una pagina

%----------------------------------------------------------
% Encabezados y pies de pagina
\usepackage{fancyhdr}
% Formato para los encabezados y pies de pagina
% En lo siguiente, fancyhead sirve para configurar la cabecera, fancyfoot para el pie.
% Justificación: C=centered, R=right, L=left, (nada)=LRC
% Página: O=odd, E=even, (nada)=OE

% Formato FANCY (Usando \thispagestyle{fancy})
\pagestyle{fancy}
% Borrar todos los ajustes previos
\fancyhf{}
% Redefinimos el comando \chaptermark (mostrar Capitulo en roman)  -> Aparece en \leftmark
%\renewcommand{\chaptermark}[1]{\markboth{#1}{}}
% Redefinimos el comando \sectionmark (mostrar Seccion en roman) -> Aparece en \rightmark
%\renewcommand{\sectionmark}[1]{\markright{#1}}
% Encabezado
\fancyhead[LE]{Valencian International University / VIU}
\fancyhead[RO]{Máster en Astronomía y Astrofísica}
% Pie de pagina
\fancyfoot[LE, RO]{\thepage}
% Tamaño de las lineas del encabezado y pie de pagina
\renewcommand{\headrulewidth}{0.5pt}
\renewcommand{\footrulewidth}{0.5pt}
% Si quitamos las lineas, mejor poner el texto de los encabezados y pie de pagina en negrita o cursiva
% Formato PLAIN (Usando \thispagestyle{plain}) CREO QUE ES EL QUE USAN LOS CHAPTERS...
\fancypagestyle{plain}{
\fancyhf{} % borrar todos los ajustes
\renewcommand{\headrulewidth}{0pt}
\renewcommand{\footrulewidth}{0.5pt}
\fancyfoot[LE, RO]{\thepage}
}
% Formato EMPTY (Usando \thispagestyle{empty})
%----------------------------------------------------------


%Paquete Theorem:
\usepackage{theorem}
%\usepackage{amsthm}
%Defs. del TEOREM:      %%%%%%%%%%%%%%%%%%%%%%%%%%%%%%%%%%%%%%
\theoremstyle{break}
\newtheorem{Def}{Definición}
\newtheorem{Prob}{Problema}
\newtheorem{Prop}[Def]{Propiedad}
\newtheorem{Propo}{Proposición}
\newtheorem{Cor}{Corolario}
\newtheorem{Teo}{Teorema}
\newtheorem{Alg}{Algoritmo}

\newcommand{\ind}{\textrm{ind}}
\newcommand{\Ind}{\textrm{Ind}}
\newcommand{\tab}{\hspace{5mm}}

%\newtheorem{Ejer}{Ejercicio}
%\newtheorem{Ejem}{Ejemplo}


{\theorembodyfont{\rmfamily} \newtheorem{Ejer}{Ejercicio}}
{\theorembodyfont{\rmfamily} \newtheorem{Ejem}{Ejemplo}}

\newenvironment{demo}{{\bf
Demostración}}{\begin{flushright}Q.E.D.\end{flushright}}
\newenvironment{nota}{{\bf Nota: }}{\vspace{12pt}}
%%%%%%%%%%%%%%%%%%%%%%%%%%%%%%%%%%%%%%%%%%%%%%%%%

% PAQUETE PARA ESPACIADOS
\usepackage{setspace}

% PARA CORREGIR
\usepackage[modulo,switch]{lineno} 

% PAQUETE PARA HACER CAPTIONS MAS VISUALES
\usepackage[font=small,format=plain,labelfont=bf,up,textfont=it,up]{caption}

% DEFINICIONES
\def\d{$\,\rm day$ }
%\def\h{$\,\rm h$ }
%\def\min{$\,\rm min$ }
%\def\seg{$\,\rm s$ }
\def\km{$\,\rm km$ }
\def\m{$\,\rm m$ }
\def\mm{$\,\rm mm$ }
\def\ppm{$\,\rm ppm$ }
\def\kg{$\,\rm Kg$ }
\def\K{$\,\rm K$ }
\def\arcsec{$\,\rm arcsec$ }
\def\pixels{$\,\rm pixels$ }
\newcommand{\cd}{{\, \rm c\, d^{-1}}}
\newcommand{\h}{{\rm ^{h}}}
\newcommand{\minutos}{{\rm ^{m}}}
\newcommand{\segundos}{{\rm ^{s}}}
\newcommand{\MHz}{\,\rm{MHz}}
\newcommand{\rmdiv}{{\rm div}}
\def\kms{$\, \rm km\, s^{-1}$}
% Comando ION
\newcommand\ion[2]{#1$\;${\scshape{#2}}}%                       % ion, i.e., CII = \ion{C}{ii}

% REVISTAS
\def\aap{Astronomy \& Astrophysics}
\def\aaps{Astronomy \& Astrophysics Supplement Series}
\def\apj{Astrophysical Journal}
\def\apjl{Astrophysical Journal Letters}
\def\apjs{Astrophysical Journal Supplement Series}
\def\mnras{Monthly Notices of the Royal Astronomical Society}
\def\nat{Nature}
\def\araa{Annual Review of Astronomy and Astrophysics}
\def\pasj{Publications of the Astronomical Society of Japan}
\def\araa{Annual Review of Astronomy \& Astrophysics}
\def\aj{Astronomical Journal}
\def\apss{Astrophysics and Space Science}
\def\pasp{Publications of the Astronomical Society of the Pacific}



% MAKES
\makeindex

% MAIN
\begin{document}
% PARA CORREGIR
%\linenumbers
% Separacion entre parrafos
\parskip 15pt

% Separacion entre lineas
\onehalfspacing
% \singlespacing
% \doublespacing



%%%%%%%%%%%%%%%%%%%%%%%%%%%%%


% TITULO DEL TRABAJO
\thispagestyle{empty}

% Título:

\begin{figure}
\begin{center}
 \includegraphics[scale=0.3]{files/figures/logo_viu.pdf}
\end{center}
\end{figure}

\begin{center}
\begin{bf}
{\large MÁSTER EN INTELIGENCIA ARTIFICIAL}\\
\end{bf}
\vspace*{1cm}
\begin{bf}
{\large DESARROLLO DE UN SISTEMA DE AUTENTICACIÓN BASADO EN LA BIOMETRÍA DEL COMPORTAMIENTO EN DISPOSITIVOS MÓVILES}\\
\end{bf}
{\large \textit{Curso 2025/26 - }}\\
\vspace{1.5 cm}
\end{center}
\begin{flushleft}
\begin{bf}
{\large Trabajo dirigido por:}\\
\end{bf}
ROBERTO ALCARAZ MACHADO\\

\end{flushleft}

\vspace*{6.5cm}
\begin{flushright}
{\bf DAVID JIMÉNEZ CASTRO}\\
\textbf{DNI:} 76052663-N\\
\textbf{e-mail:} \href{djimenezc1@student.universidadviu.com}{djimenezc1@student.universidadviu.com}
\end{flushright}


\newpage

\newpage


% INDICE
\pagestyle{fancy}
\pagenumbering{roman}
\tableofcontents


% CAPITULOS
\newpage
\pagestyle{fancy}
\pagenumbering{arabic}

% CAPITULO 1
\chapter{Resumen}
\label{chapter:01}




Esto es una introducción al Capítulo \ref{chapter:01}. Acabamos de introducir una referencia cruzada. Bla bla bla bla bla bla bla bla bla bla bla bla bla bla bla bla bla bla bla bla bla bla bla bla bla bla bla bla bla bla bla bla bla bla bla bla bla bla bla bla bla bla bla bla bla bla bla bla bla bla bla bla bla bla bla bla bla bla bla bla bla bla bla bla bla bla bla bla bla bla bla bla bla bla bla bla bla bla bla bla bla bla bla bla bla bla bla bla bla bla bla bla bla bla bla bla bla bla bla bla bla bla bla bla bla bla bla bla bla bla bla bla bla bla bla bla bla bla bla bla bla bla bla bla bla bla bla bla bla bla bla bla bla bla bla bla bla bla bla bla bla bla bla bla bla bla bla bla bla bla bla bla bla bla bla bla bla bla bla bla bla bla bla bla bla bla bla bla bla bla bla bla bla bla bla bla bla bla bla bla bla bla bla bla bla bla bla bla bla bla bla bla bla bla bla bla bla bla bla bla bla bla bla bla bla bla bla bla bla bla bla bla bla bla bla bla bla bla bla bla bla bla bla bla bla bla bla bla bla bla bla bla bla bla bla bla bla bla bla bla bla bla bla bla bla bla bla bla bla bla bla bla bla bla bla bla bla bla bla bla bla bla bla bla bla bla bla bla bla bla bla bla bla bla bla bla bla bla bla bla bla bla bla bla bla bla bla bla bla bla bla bla bla bla bla bla bla bla bla bla bla bla bla bla bla bla bla bla bla bla bla bla bla bla bla bla bla bla bla bla bla bla bla bla bla bla bla bla bla bla bla bla bla bla bla bla bla bla bla bla bla bla bla bla bla bla bla bla bla bla bla bla bla bla bla bla bla bla bla bla bla bla bla bla bla bla bla bla bla bla bla bla bla bla bla bla bla bla bla bla bla bla bla bla bla bla bla bla bla bla bla bla bla bla bla bla bla bla bla bla bla bla bla bla bla bla bla bla bla bla bla bla bla bla bla bla bla bla bla bla bla bla bla bla bla bla bla bla bla bla bla bla bla bla bla bla bla bla bla bla bla bla bla bla bla bla bla bla bla bla bla bla bla bla bla bla bla bla bla bla bla bla bla bla bla bla bla bla bla bla bla bla bla bla bla bla bla bla bla bla bla bla bla bla bla bla bla bla bla bla bla bla 


% CAPITULO 2
\chapter{Introducción}
\label{chapter:02}

En el panorama actual de la ciberseguridad móvil, la dependencia de métodos de autenticación estática ---como contraseñas, PINs o patrones de dibujo--- presenta vulnerabilidades críticas frente a ataques de ingeniería social e interceptación. Si bien la biometría fisiológica (huella dactilar o reconocimiento facial) ha reforzado el perímetro de seguridad, estos métodos actúan como guardianes de ``un solo paso'' que no protegen la sesión una vez iniciada.

El presente proyecto propone el desarrollo de un \textbf{Sistema de Autenticación Basado en la Biometría del Comportamiento}, un enfoque que permite la verificación de identidad de forma \textbf{pasiva, continua y transparente} para el usuario.

\section{El Fundamento de la Biometría Conductual}

A diferencia de los rasgos físicos, la biometría conductual se centra en \textbf{cómo} interactúa el usuario con su dispositivo. Este sistema aprovecha la singularidad de los patrones neurofisiológicos y motores, analizando variables clave capturadas de manera transparente:

\begin{itemize}
    \item \textbf{Dinámica de Tecleo (\textit{Keystroke Dynamics}):} Estudio de los tiempos de presión (\textit{Dwell Time}), las latencias entre teclas (\textit{Flight Time}) y los intervalos entre pulsaciones (\textit{Inter-key Interval}).
    \item \textbf{Interacción Táctil:} Patrones de desplazamiento (\textit{scrolling}), toques en pantalla (\textit{tapping}) y gestos en la interfaz.
    \item \textbf{Sensores de Movimiento (\textit{IMU}):} Uso de acelerómetros, giroscopios y magnetómetros para capturar la micro-gestualidad y el manejo físico del terminal durante la interacción.
\end{itemize}

\section{Innovación Tecnológica y Arquitectura}

Para superar las limitaciones de los modelos clásicos (como SVM o KNN), que presentan dificultades de escalabilidad en entornos masivos, este sistema se fundamenta en arquitecturas de \textbf{Aprendizaje Profundo (\textit{Deep Learning})}:

\begin{itemize}
    \item \textbf{Modelos Híbridos y Atención:} Integración de redes neuronales convolucionales (CNN) para la extracción de características locales y redes recurrentes (LSTM) para dependencias temporales globales.
    \item \textbf{Mecanismos de Atención Dual:} Empleo de arquitecturas basadas en \textit{transformers} (como \textit{STDAT}) que aplican atención tanto temporal como de canal para capturar patrones característicos.
    \item \textbf{Fusión Multimodal:} Combinación de la dinámica de tecleo con datos de sensores inerciales para incrementar la robustez del sistema frente a cambios de contexto.
    \item \textbf{Funciones de Pérdida Avanzadas:} Implementación de técnicas como \textit{Triplet Loss} y \textit{Set2set Loss}, permitiendo comparar flujos continuos de datos contra el perfil del usuario con mínima fricción.
\end{itemize}



% CAPITULO 3
\chapter{Estado del Arte}
\label{chapter:03}
El estado del arte en la biometría del comportamiento en dispositivos móviles abarca una amplia gama de técnicas y 
metodologías que buscan identificar y autenticar a los usuarios basándose en sus patrones de interacción con el 
dispositivo. 
A continuación, se presentan algunas de las investigaciones y desarrollos más relevantes en este campo. 

\section{IJCB 2022 MobileB2C: Competición de Autenticación Basada en Comportamiento Móvil}


El trabajo de \cite{stragapede2022ijcb2022mobilebehavioral} presenta una \textbf{evaluación comparativa} de sistemas de autenticación móvil basados en \textbf{biometría conductual}, capturada de manera transparente mientras el usuario interactúa con su dispositivo.

\subsubsection{Base de Datos Utilizada}

El estudio emplea la base pública \textbf{BehavePassDB}, recopilada en condiciones reales, que incluye:
\begin{itemize}
    \item Dinámica de tecleo (\textit{keystroke})
    \item Lectura de texto (\textit{text reading})
    \item Deslizamiento de galería (\textit{gallery swiping})
    \item Toques en pantalla (\textit{tapping})
    \item Sensores como acelerómetro, giroscopio, magnetómetro, entre otros
\end{itemize}
Para el \textit{benchmarking}, además de la identeficación del usuario legítimo, se consideran dos tipos de ataques de impostores:
\begin{itemize}
    \item \textbf{Impostores aleatorios}: otro usuario con un dispositivo distinto.
    \item \textbf{Impostores hábiles}: individuos que intentan imitar al usuario legítimo.
\end{itemize}
\subsubsection{Conclusiones}

Los resultados de la competición \textbf{MobileB2C} muestran que la autenticación basada en comportamiento es \textbf{viable}, aunque sigue siendo un desafío complejo debido a la variabilidad del entorno y la dificultad de modelar múltiples modalidades de interacción.

Como contribución adicional, el estudio consolida \textbf{MobileB2C como una competición continua}, proporcionando una base de datos abierta y un protocolo estándar que facilita nuevas investigaciones en autenticación conductual bajo condiciones realistas.

{\centering
\captionof{table}{Resultados durante la fase de evaluación (AUC [\%])}
\label{TablaB2C}
\begin{tabular}{c l c c c}
    \toprule
    \# & Team & Mixed & Random & Skilled \\
    \midrule
    \multicolumn{5}{c}{\textbf{Task 1: Keystroke}} \\
    \midrule
    1 & NUS-UoA-UoM & \textbf{66.37} & \textbf{64.77} & \textbf{67.91} \\
    2 & HCI Essen   & 51.12 & 53.02 & 51.23 \\
    3 & HBKU CS Lab & 51.25 & 49.38 & 53.13 \\
    4 & JAIRG       & 45.57 & 52.29 & 39.89 \\
    \midrule
    \multicolumn{5}{c}{\textbf{Task 2: Text Reading}} \\
    \midrule
    1 & HCI Essen   & \textbf{57.63} & \textbf{61.27} & \textbf{53.98} \\
    2 & NUS-UoA-UoM & 54.89 & 58.49 & 51.29 \\
    3 & JAIRG       & 50.63 & 50.00 & 41.25 \\
    4 & HBKU CS Lab & 48.27 & 59.42 & 37.13 \\
    \midrule
    \multicolumn{5}{c}{\textbf{Task 3: Gallery Swiping}} \\
    \midrule
    1 & HBKU CS Lab & \textbf{61.54} & \textbf{67.35} & \textbf{55.73} \\
    2 & JAIRG       & 55.94 & 61.95 & 50.62 \\
    3 & NUS-UoA-UoM & 55.66 & 55.54 & 55.77 \\
    4 & HCI Essen   & 54.72 & 57.30 & 51.17 \\
    \midrule
    \multicolumn{5}{c}{\textbf{Task 4: Tapping}} \\
    \midrule
    1 & HBKU CS Lab & \textbf{59.58} & \textbf{57.22} & \textbf{61.94} \\
    2 & NUS-UoA-UoM & 52.39 & 54.72 & 50.06 \\
    3 & JAIRG       & 46.25 & 48.75 & 43.75 \\
    4 & HCI Essen   & 43.89 & 40.16 & 47.62 \\
    \bottomrule
\end{tabular}
\par
}

\section{BehaveFormer: A Framework with Spatio-Temporal Dual Attention
Transformers for IMU enhanced Keystroke Dynamics}


El trabajo de \cite{senerath2023} propone \textit{BehaveFormer}, un sistema de autenticación continua en dispositivos móviles basado en la biometría del comportamiento. El enfoque combina dinámicas de tecleo con datos procedentes de sensores inerciales (IMU), presentes de forma estándar en la mayoría de teléfonos inteligentes. El núcleo del modelo es el \textit{Spatio-Temporal Dual Attention Transformer} (STDAT), una arquitectura basada en \textit{transformers} que emplea mecanismos de atención tanto temporal como de canal para capturar patrones característicos en el comportamiento del usuario.

Para la modalidad de tecleo, se utilizan secuencias de di-gramas y tri-gramas enriquecidas con tiempos de pulsación (\textit{hold}), transiciones entre eventos y latencias entre teclas. En el caso de la IMU, se extraen derivadas de primer y segundo orden sobre los ejes tridimensionales y se aplica la transformada rápida de Fourier (FFT), obteniendo un vector descriptivo de 36 características por instante.

Cada modalidad es procesada por un STDAT independiente; posteriormente ambas representaciones se fusionan mediante concatenación para generar una incrustación final del usuario. El entrenamiento se realiza mediante \textit{triplet loss}, favoreciendo que muestras del mismo usuario queden próximas en el espacio de representación y que las de distintos usuarios estén separadas.

El método se evalúa sobre tres conjuntos de datos ampliamente utilizados en autenticación continua: Aalto DB, HMOG DB y HuMIdb. Los resultados muestran mejoras significativas frente al estado del arte previo, alcanzando por ejemplo un EER del 1.80\% utilizando solo tecleo en Aalto DB, y un EER del 2.95\% al combinar tecleo e IMU en HuMIdb. Estas cifras confirman la eficacia de la fusión multimodal y del mecanismo de atención dual.

En conjunto, BehaveFormer demuestra que la combinación de información de tecleo y sensores inerciales, unida a arquitecturas basadas en transformers, constituye una vía sólida para sistemas de autenticación pasiva y continua, incrementando la seguridad sin exigir interacción explícita del usuario.

\section{Type2Branch: Keystroke Biometrics based on a Dual-branch Architecture
with Attention Mechanisms and Set2set Loss}

En esta sección se analiza el artículo de \cite{type2branch} que introduce \textit{Type2Branch}, un sistema de autenticación basado en la dinámica de tecleo que se destaca por su capacidad de generalización en escenarios reales, superando las limitaciones de modelos anteriores como \textit{TypeNet}.

\subsection{Evolución de las Arquitecturas de Aprendizaje Profundo}
Históricamente, los sistemas de \textit{Keystroke Dynamics} se basaban en redes recurrentes simples para modelar secuencias temporales. Sin embargo, las limitaciones de estas para capturar patrones locales llevaron al desarrollo de enfoques híbridos:

\begin{itemize}
    \item \textbf{Arquitectura Dual-Branch:} El modelo propone una rama de Redes Neuronales Convolucionales (CNN) para la extracción de características locales y una rama de Redes Recurrentes (RNN/LSTM) para las dependencias temporales globales.
    \item \textbf{Mecanismos de Atención:} A diferencia de los modelos precedentes como \textit{TypeNet}, se introducen capas de atención que permiten al sistema ponderar eventos de pulsación específicos que poseen mayor carga discriminativa para la identidad del usuario.
\end{itemize}

\subsection{Funciones de Pérdida y Verificación}
Uno de los mayores avances del artículo es la transición de funciones de pérdida punto a punto hacia enfoques de conjuntos:

\begin{itemize}
    \item \textbf{Set2set Loss:} Frente al tradicional \textit{Triplet Loss}, la función \textit{Set2set} permite comparar un conjunto de muestras de enrolamiento contra una muestra de consulta. Esto es crítico para la \textbf{autenticación pasiva}, donde la decisión se basa en el flujo continuo de datos y no en una única entrada estática.
\end{itemize}

\subsection{Métricas de Rendimiento en Autenticación Pasiva}
El rendimiento reportado establece un nuevo estándar para escenarios de texto libre (\textit{free-text}) y dispositivos heterogéneos:

\begin{table}[h]
\centering
\caption{Rendimiento de Type2Branch en entornos reales.}
\begin{tabular}{l|c|c|r}
\hline
\textbf{Escenario} & \textbf{Usuarios} & \textbf{EER (\%)} & \textbf{Longitud de Secuencia} \\
\hline
Desktop (Escritorio) & 15,000 & $0.77\%$ & 50 caracteres \\
Mobile (Táctil) & 5,000 & $1.03\%$ & 50 caracteres \\
\hline
\end{tabular}
\end{table}

\subsection{Importancia para la Autenticación Pasiva}
El modelo demuestra que es posible alcanzar una alta tasa de precisión con ráfagas cortas de actividad (50 caracteres). En el contexto de la seguridad continua, esto reduce el \textit{Time-to-Detection} (TTD) de un posible impostor, permitiendo una revocación de acceso casi instantánea sin intervención del usuario.

\section{Estado del Arte: Autenticación Continua y Ensamblados de Redes Profundas}

El estudio de la biometría de teclado (\textit{Keystroke Dynamics}) ha transitado desde el análisis de textos fijos hacia la autenticación pasiva y continua en entornos de texto libre. El trabajo de \cite{deepkeystroke2022} contextualiza este avance mediante la comparación de técnicas tradicionales frente a arquitecturas de aprendizaje profundo.

\subsection{Evolución de las Técnicas de Clasificación}
La literatura previa se divide fundamentalmente en dos vertientes metodológicas:

\begin{itemize}
    \item \textbf{Aproximaciones Clásicas:} Se basan en algoritmos de aprendizaje automático supervisado como \textit{K-Nearest Neighbors} (KNN), \textit{Naive Bayes} y \textit{Support Vector Machines} (SVM). Aunque eficaces en datasets pequeños, presentan limitaciones de escalabilidad y precisión en escenarios de autenticación continua donde el flujo de datos es masivo y desestructurado.
    \item \textbf{Aprendizaje Profundo (Deep Learning):} El estado del arte reciente destaca el uso de Redes Neuronales Convolucionales (CNN) para extraer características espaciales de las pulsaciones y Redes Neuronales Recurrentes (RNN), específicamente LSTM, para capturar la dependencia temporal. Aversano et al. introducen el concepto de \textit{Ensemble Learning}, combinando múltiples clasificadores base para reducir la varianza y mejorar la robustez del sistema.
\end{itemize}

\subsection{Extracción de Características y Representación de Datos}
El consenso en las investigaciones actuales identifica tres métricas temporales críticas para la biometría conductual:
\begin{enumerate}
    \item \textbf{Dwell Time (DT):} El tiempo que una tecla permanece presionada.
    \item \textbf{Flight Time (FT):} El intervalo entre la liberación de una tecla y la pulsación de la siguiente.
    \item \textbf{Inter-key Interval (IKI):} El tiempo transcurrido entre dos pulsaciones consecutivas.
\end{enumerate}
El artículo subraya que la integración de estos rasgos en arquitecturas profundas permite omitir la ingeniería de características manual, ya que la red aprende representaciones jerárquicas de los patrones de tecleo.

\subsection{Desafíos en la Autenticación Pasiva}
A pesar de los avances, el estado del arte identifica brechas significativas que este estudio busca solventar:
\begin{itemize}
    \item \textbf{Fragmentación de Datos:} La mayoría de los estudios previos utilizan datasets pequeños (menos de 100 usuarios). El artículo destaca la necesidad de \textit{benchmarks} masivos, proponiendo un dataset integrado de más de 160,000 usuarios.
    \item \textbf{Generalización:} La dificultad de los modelos para mantener la precisión cuando el usuario cambia de contexto de escritura o de dispositivo.
    \item \textbf{Toma de Decisión Continua:} La transición de una autenticación de un solo paso (login) a una monitorización constante sin fricción para el usuario.
\end{itemize}

\section*{Estado del Arte: Biometría Conductual en Dispositivos Móviles}

El artículo de \cite{Stragapede2022} sitúa su investigación en el contexto de la autenticación pasiva mediante biometría conductual en smartphones, analizando tanto la interacción táctil como los sensores de movimiento (background sensors).

\subsection*{Clasificación de Técnicas Anteriores}
Los autores clasifican los trabajos previos principalmente en base a las modalidades biométricas y las arquitecturas de procesamiento:
\begin{itemize}
    \item \textbf{Dinámica de Tecleo (Keystroke Dynamics):} Se divide en escenarios de \textit{texto fijo} (contraseñas) y \textit{texto libre}. Se mencionan aproximaciones basadas en redes neuronales recurrentes, específicamente Long Short-Term Memory (LSTM) para autenticación a gran escala.
    \item \textbf{Gestos Táctiles (Touch Gestures):} Técnicas que analizan el desplazamiento (scrolling), toques (tapping) y dibujo de patrones.
    \item \textbf{Sensores de Movimiento (Background Sensors):} Uso de acelerómetros, giroscopios y magnetómetros para capturar patrones físicos del usuario durante la interacción.
    \item \textbf{Arquitecturas de Aprendizaje:} El artículo distingue entre modelos estadísticos tradicionales y enfoques modernos de Deep Learning (como CNNs para gestos y LSTMs para secuencias temporales).
\end{itemize}

\subsection*{Limitaciones Identificadas}
El estudio subraya varias deficiencias en la literatura actual:
\begin{enumerate}
    \item \textbf{Fragmentación de Modalidades:} La mayoría de los estudios se centran en una sola modalidad (solo tecleo o solo acelerómetro), perdiendo la sinergia de la fusión multimodal.
    \item \textbf{Escalabilidad y Realismo:} Muchos trabajos utilizan bases de datos pequeñas o recolectadas en entornos de laboratorio controlados que no reflejan el uso cotidiano ("in-the-wild").
    \item \textbf{Ventanas de Tiempo Elevadas:} Algunos sistemas requieren periodos de observación demasiado largos para alcanzar precisiones aceptables, lo que reduce la eficacia de la autenticación pasiva inmediata.
\end{enumerate}

\subsection*{Problema Específico a Resolver}
El vacío que este artículo intenta llenar es la \textbf{evaluación exhaustiva de la fusión de múltiples biometrías conductuales} (tecleo, scroll, dibujo, sensores de fondo) bajo un protocolo común y utilizando una base de datos pública de gran escala (\textit{HuMIdb}). El objetivo es determinar cuál es la combinación mínima de sensores y tiempo de análisis necesaria para lograr una autenticación robusta y transparente.


% CAPITULO 4
\chapter{Marco Teórico}
\label{chapter:04}
\section{Marco Teórico}
\subsection{Fundamentos de autenticación y seguridad}
\subsection{Modalidades de biometría conductual}
\subsection{Métodos de aprendizaje automático aplicados}
\subsection{Fusión multimodal}
\subsection{Funciones de pérdida para verificación biométrica}
\subsection{Bases de datos y benchmarks del dominio}
\subsection{Privacidad y consideraciones éticas}

% CAPITULO 5
\chapter{Metodología}
\label{chapter:05}
\chapter{Desarrollo del Proyecto}\label{chap:desarrollo}
\section{Descripción del Problema}

El problema central de este trabajo es la autenticación continua y pasiva de usuarios en dispositivos móviles 
a partir de su biometría conductual. A diferencia de los mecanismos de autenticación estática —contraseñas, 
PINs o biometría fisiológica— que únicamente verifican la identidad en el momento del acceso, el objetivo es 
mantener una verificación activa durante toda la sesión sin que el usuario deba realizar ninguna acción adicional.
Este enfoque plantea un problema de \textbf{verificación biométrica de una clase} (\textit{one-class} o \textit{open-set}): 
dado un perfil conductual enrolado para un usuario legítimo, el sistema debe decidir, a partir de una ventana 
corta de interacción, si el individuo que está operando el dispositivo corresponde a dicho perfil o se trata 
de un impostor. La dificultad inherente reside en que durante la fase de entrenamiento no se dispone de muestras 
etiquetadas de todos los posibles impostores, lo que impide formular el problema como una clasificación cerrada 
convencional.
Las modalidades conductuales seleccionadas para este trabajo son tres, elegidas por su complementariedad y por
la riqueza de información identitaria que aportan:

\begin{itemize}
    \item \textbf{Dinámica de tecleo (\textit{keystroke dynamics})}: secuencias de tiempos de pulsación (\textit{Dwell Time}, DT),
     tiempos de vuelo entre teclas (\textit{Flight Time}, FT) e intervalos entre pulsaciones consecutivas 
     (\textit{Inter-key Interval}, IKI). Constituye la modalidad más estudiada en la literatura y ofrece alta 
     discriminabilidad incluso con ráfagas cortas de texto.
    \item \textbf{Dinámica de firma (\textit{signature dynamics})}: patrones de presión, velocidad y
     aceleración durante el trazo de la firma en pantalla táctil. Aporta una señal de alta varianza 
     inter-usuario y baja varianza intra-usuario cuando el individuo firma de forma natural.
    \item \textbf{Gestos de desplazamiento (\textit{scroll dynamics})}: características extraídas de 
    los movimientos de deslizamiento sobre la pantalla, incluyendo velocidad, aceleración, longitud del trazo 
    y ángulo de desplazamiento. Su inclusión está condicionada al análisis del beneficio marginal que aporta en 
    la fusión multimodal.
\end{itemize}

El principal desafío técnico es la \textbf{heterogeneidad temporal} de las señales: la dinámica de tecleo 
es discreta y de longitud variable, la firma es una señal continua de corta duración, y el scroll produce 
eventos esporádicos dependientes del contexto de uso. El sistema debe ser capaz de integrar estas tres fuentes 
de forma coherente y tomar decisiones de autenticación con latencia mínima, idealmente tras menos de 50 eventos
de interacción.

\section{Metodología}\label{sec:metodologia}
Para abordar el problema descrito, se propone una metodología experimental comparativa basada en dos 
enfoques arquitectónicos complementarios, cuyo rendimiento se evaluará bajo un protocolo común.

\subsection{Preprocesamiento y Extracción de Características}
Previo al modelado, cada modalidad requiere un pipeline de preprocesamiento específico:

\begin{itemize}
    \item \textbf{Tecleo}: a partir de los eventos de pulsación se extraen las tripletas $(DT_i, FT_i, IKI_i)$ para cada tecla 
    $i$ de la secuencia. Las secuencias se segmentan en ventanas deslizantes de longitud fija $N$ 
    (objetivo: $N = 50$ eventos), con solapamiento configurable para la autenticación continua. Los 
    valores atípicos se filtran mediante umbrales basados en percentiles y las secuencias se normalizan
     con estadísticos calculados por usuario en la fase de enrolamiento.
    \item \textbf{Firma}: se extraen series temporales de posición $(x_t, y_t)$, presión $p_t$ y sus derivadas 
    de primer y segundo orden. La señal se remuestrea a una frecuencia fija para garantizar la homogeneidad 
    de la representación de entrada.
    \item \textbf{Scroll}: se caracterizan cada gesto de deslizamiento mediante un vector de rasgos agregados: 
    velocidad media, velocidad máxima, aceleración media, longitud euclidiana del trazo, duración y desviación 
    angular respecto al eje vertical.
\end{itemize}

\subsection{Enfoque 1: Arquitectura basada en Transformer}

El primer enfoque propone una arquitectura \textit{end-to-end} inspirada en BehaveFormer~\cite{senerath2023behaveformer}
 y Type2Branch~\cite{gonzalez2025type2branch}, que procesa las tres modalidades de forma conjunta 
 mediante mecanismos de atención.
Cada secuencia de interacción se trata como una serie de \textit{tokens} de entrada. Un codificador 
Transformer independiente por modalidad —con capas de \textit{Multi-Head Self-Attention} y redes 
\textit{feed-forward} posicionales— genera una representación de contexto para cada fuente de señal. Las 
representaciones resultantes se fusionan mediante un módulo de \textbf{atención cruzada entre modalidades} 
(\textit{cross-modal attention}), que aprende a ponderar dinámicamente la contribución de cada fuente en función del contexto de uso.
La función de pérdida empleada para el entrenamiento es \textit{Set2set Loss}~\cite{gonzalez2025type2branch}, 
que compara un conjunto de muestras de enrolamiento $\mathcal{S} = \{s_1, \ldots, s_k\}$ contra una muestra de consulta $q$, 
permitiendo una estimación más robusta de la similitud que el clásico \textit{Triplet Loss} puntual:

\begin{equation}
    \mathcal{L}_{\text{set2set}} = \max\left(0,\; m + d\!\left(\bar{e}_{\mathcal{S}},\, q^+\right) - d\!\left(\bar{e}_{\mathcal{S}},\, q^-\right)\right)
    \label{eq:set2set}
\end{equation}

donde $\bar{e}_{\mathcal{S}}$ es la representación agregada del conjunto de enrolamiento, 
$q^+$ una muestra del usuario legítimo, $q^-$ una muestra de impostor y $m$ el margen de separación.

\subsection{Enfoque 2: Ensemble de Modelos Especializados por Modalidad}

El segundo enfoque adopta una estrategia modular: se entrena un modelo especializado e independiente 
para cada modalidad conductual, y sus puntuaciones de similitud se combinan mediante una estrategia de fusión 
a nivel de \textit{score}.

\begin{itemize}
    \item \textbf{Modelo de tecleo}: red LSTM bidireccional seguida de una capa de proyección métrica, 
    entrenada con \textit{Triplet Loss}, siguiendo la línea de TypeNet~\cite{acien2022typenet}.
    \item \textbf{Modelo de firma}: red CNN-LSTM que extrae características locales de los segmentos del 
    trazo (CNN) y modela la dinámica temporal global (LSTM).
    \item \textbf{Modelo de scroll}: clasificador ligero basado en un perceptrón multicapa (MLP) o una red 
    CNN-1D sobre los vectores de rasgos agregados por gesto.
\end{itemize}

La fusión de puntuaciones se explorará mediante tres estrategias: combinación lineal con pesos fijos, 
combinación lineal con pesos aprendidos, y un meta-clasificador entrenado sobre las puntuaciones de los 
modelos base.

\subsection{Protocolo de Evaluación}

Ambos enfoques se evaluarán bajo un protocolo común de verificación, siguiendo las directrices establecidas en 
las competiciones MobileB2C~\cite{stragapede2022mobileb2c}:

\begin{itemize}
    \item \textbf{Partición de datos}: división en conjuntos de enrolamiento, desarrollo y evaluación, 
    sin solapamiento de usuarios entre particiones.
    \item \textbf{Escenarios de impostor}: se considerarán impostores aleatorios (usuarios distintos sin 
    conocimiento del objetivo) e impostores hábiles (usuarios que intentan imitar al legítimo), cuando el 
    dataset seleccionado lo permita.
    \item \textbf{Métricas principales}: \textit{Equal Error Rate} (EER) como métrica primaria de comparación, 
    complementada con AUC, FAR@1\%FRR y el \textit{Time-to-Detection} (TTD) estimado.
\end{itemize}


\section{Resultados Preliminares}\label{sec:resultados_preliminares}

En el momento de redacción de este documento, el desarrollo del sistema se encuentra en fase de diseño 
arquitectónico y planificación experimental. Por tanto, los resultados que se presentan a continuación son 
estimaciones de referencia (\textit{baseline}) derivadas de la replicación parcial de trabajos del estado del 
arte sobre datasets públicos, con el objetivo de establecer un punto de comparación para las arquitecturas 
propuestas.

\subsection{Baseline de Referencia}

A continuación se muestra la tabla que recoge los valores de EER reportados por los sistemas más relevantes del estado
 del arte sobre las modalidades y datasets que se emplearán en este trabajo,
  y que constituyen el umbral de rendimiento que el sistema propuesto deberá superar.

\begin{table}[h]
\centering
\caption{Resultados de referencia del estado del arte (EER~\%).}\label{tab:baseline}
\begin{tabular}{llccc}
\textbf{Sistema} & \textbf{Modalidad} & \textbf{Dataset} & \textbf{EER (\%)} & \textbf{Seq. length} \\
TypeNet~\cite{acien2022typenet}             & Tecleo              & Aalto DB   & 2.20 & 50 chars \\
Type2Branch~\cite{gonzalez2025type2branch}  & Tecleo              & Mobile     & 1.03 & 50 chars \\
BehaveFormer~\cite{senerath2023behaveformer}& Tecleo + IMU        & HuMIdb     & 2.95 & ---      \\
MobileB2C top-1~\cite{stragapede2022mobileb2c} & Keystroke      & BehavePassDB & \multicolumn{1}{c}{AUC: 66.37} & --- \\
\end{tabular}
\end{table}

\subsection{Estimaciones Esperadas del Sistema Propuesto}

A modo orientativo, y en base a las mejoras arquitectónicas propuestas respecto a los sistemas de referencia,
 se anticipan los rangos de rendimiento recogidos en la siguiente tabla:

\begin{table}[h]
\centering
\caption{Estimación de rendimiento esperado del sistema propuesto (EER~\%). 
\textbf{[PENDIENTE DE ACTUALIZAR]}}\label{tab:resultados_esperados}
\begin{tabular}{llcc}
\textbf{Enfoque} & \textbf{Modalidades} & \textbf{EER esperado (\%)} & \textbf{Observaciones} \\
Transformer unificado  & Tecleo + Firma          & $\sim$1.5 -- 2.5  & Fusión cross-modal \\
Transformer unificado  & Tecleo + Firma + Scroll & $\sim$1.2 -- 2.0  & Mejora con scroll  \\
Ensemble modular       & Tecleo + Firma          & $\sim$1.8 -- 3.0  & Fusión de scores   \\
Ensemble modular       & Tecleo + Firma + Scroll & $\sim$1.5 -- 2.5  & Meta-clasificador  \\
\end{tabular}
\end{table}

Estas estimaciones se fundamentan en la hipótesis de que la fusión multimodal de tecleo y firma 
aportará una mejora consistente respecto al uso de tecleo en solitario, tal y como sugieren los trabajos 
previos en fusión de modalidades conductuales~\cite{stragapede2022patterns}. 
La incorporación del scroll se plantea como un experimento ablativo para cuantificar su contribución marginal 
al rendimiento global del sistema.

% CAPITULO 6
\chapter{Resultados}
\label{chapter:06}
\input{./files/cap06.tex}

% CAPITULO 7
\chapter{Conclusiones}
\label{chapter:07}
\input{./files/cap07.tex}

% CAPITULO 8
\chapter{Limitaciones y trabajos futuros}
\label{chapter:08}
\input{./files/cap08.tex}

% BIBLIOGRAFIA
\bibliographystyle{./sty/astron}
\bibliography{./biblio/tfm_biblio}

%%%%%%%%%%%%%%%%%%%%%%%%%%%%%

\end{document}
